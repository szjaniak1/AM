\documentclass{article}
\usepackage[T1]{fontenc}
\usepackage{graphicx}
\usepackage{subcaption}
\usepackage{sectsty}
\usepackage{multirow}
\usepackage[a4paper, total={6in, 8in}]{geometry}

\def\v{0.4}

\title{%
	Algorytmy Metahuerystyczne \\
	\large Lista 2}
\author{Szymon Janiak}
\begin{document}
\maketitle

\section*{Opis problemu}
	Testujemy metaheurystykę Tabu Search oraz Symulowanego wyżarzania.

\subsection*{Porównanie wyników dla rozwiązań bazujących na MST oraz losowej permutacji}
    \begin{center}
    \resizebox{8cm}{!}{
        \begin{tabular}{|c||c|c|}
        \hline
            & MST & losowa permutacja\\
            \hline\hline
            xqg237 & 1526 & 3048\\
            \hline
            pbl395 & 2523 & 5152\\
            \hline
           	pbm436 & 3013 & 6470\\
        \hline
		\end{tabular}}
		\\Waga najlepszych uzyskanych rozwiązań dla Symulowanego wyżarzania.
    \end{center}
\subsection*{Wnioski}
	Widzimy, że bazowanie na MST znacząco poprawia jakość rozwiązań już dla bardzo małych danych. Dysproporcja zwiększa się wraz ze zwiększaniem ilości wierzchołków.

\subsection*{Dobór parametrów}
	\begin{itemize}
		\item N - liczba wierzchołków
	\end{itemize}
\subsubsection*{Tabu Search}
	\begin{itemize}
		\item Temperatura początkowa: $initial\_temp = N \cdot \alpha$, $\alpha = 0.85$
		\item Chłodzenie: $temp = temp \cdot \beta $, $\beta = 0.85$
		\item Długość epoki: $epoch\_range = initial\_temp \cdot \delta = 0.7$
		\item Liczba iteracji: $max\_it = N \cdot \delta$, $\gamma = 0.7$
		\item Typ otoczenia: \texttt{SWAP}
		\item Rozwiązanie początkowe: oparte o MST
	\end{itemize}

\subsubsection*{Symulowane wyżarzanie}
	\begin{itemize}
		\item Długość listy tabu: $t\_size = N \cdot \alpha$, $ \alpha = 0.1$
		\item Liczba iteracji: $max\_it = N \cdot \beta$, $\beta = 0.2$
		\item Typ otoczenia: \texttt{SWAP}
		\item Wybór otoczenia: pełne
		\item Rozwiązanie początkowe: oparte o MST
	\end{itemize}

\subsection*{Wyniki}
\begin{table}[h!]
    \centering
    \begin{tabular}{|c|c|c|c|c|c|}
        \hline
        \multirow{2}{*}{Przykład} & \multirow{2}{*}{Optymalna trasa} & \multicolumn{2}{|c|}{Sumulowane Wyżarzanie}  & \multicolumn{2}{|c|}{Tabu Search}  \\
        \cline{3-6}
        & & śr. waga & min. waga & śr. waga & min. waga \\
        \hline
        xqf131 & 564 & 1023 & 902 & 976 & 863 \\
        \hline
        xqg237 & 1019 & 1756 & 1526 & 1689 & 1423 \\
        \hline
        pma343 & 1368 & 2395 & 2110 & 2154 & 1922 \\
        \hline
        pka379 & 1332 & 2380 & 2252 & 2298 & 2079 \\
        \hline
        bcl380 & 1621 & 2630 & 2437 & 2350 & 2221 \\
        \hline
        pbl395 & 1281 & 2863 & 2523 & 2577 & 2349 \\
        \hline
        pbk411 & 1343 & 3031 & 2761 & 2733 & 2589 \\
        \hline
        pbn423 & 1365 & 3257 & 2857 & 2868 & 2620 \\
        \hline
        pbm436 & 1443 & 3340 & 3013 & 2963 & 2865 \\
        \hline
        xql662 & 2513 & 4382 & 3903 & 4192 & 3891 \\
        \hline
        xit1083 & 3558 & 5125 & 4946 & 4909 & 4768 \\
        \hline
        icw1483 & 4416 & 6531 & 6228 & 6339 & 6139 \\
        \hline
        djc1785 & 6115 & 7943 & 7660 & 7570 & 7450 \\
        \hline
        dcb2086 & 6600 & 8629 & 8521 & 8471 & 8257 \\
        \hline
        pds2566 & 7643 & 10131 & 9856 & 9429 & 9263 \\
        \hline
    \end{tabular}
    \caption{Porównanie wyników metaheurystyk: Symulowane Wyżarzanie oraz Tabu Search.}
\end{table}

\end{document}