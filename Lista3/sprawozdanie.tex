\documentclass{article}
\usepackage[T1]{fontenc}
\usepackage{graphicx}
\usepackage{subcaption}
\usepackage{sectsty}
\usepackage{multirow}
\usepackage[a4paper, total={6in, 8in}]{geometry}

\def\v{0.4}

\title{%
	Algorytmy Metahuerystyczne \\
	\large Lista 2}
\author{Szymon Janiak}
\begin{document}
\maketitle

\section*{Opis problemu}
	Testujemy algorytm Tabu Search oraz algorytm symulowanego wyżarzania.

\subsection*{Porównanie wyników dla rozwiązań bazujących na MST oraz losowej permutacji}
    \begin{center}
    \resizebox{8cm}{!}{
        \begin{tabular}{|c||c|c|}
        \hline
            & MST & losowa permutacja\\
            \hline\hline
            xqg237 & 1526 & 3048\\
            \hline
            pbl395 & 1883 & 4152\\
            \hline
           	pbm436 & 2112 & 4870\\
        \hline
		\end{tabular}}
		\\Waga najlepszych uzyskanych rozwiązań dla algorytmu symulowanego wyżarzania.
    \end{center}
\subsection*{Wnioski}
	Widzimy, że bazowanie na MST znacząco poprawia jakość rozwiązań już dla bardzo małych danych. Dysproporcja zwiększa się wraz ze zwiększaniem ilości wierzchołków.

\subsection*{Dobór parametrów}
	\begin{itemize}
		\item N - liczba wierzchołków
	\end{itemize}
\subsubsection*{Tabu Search}
	\begin{itemize}
		\item Temperatura początkowa: $initial\_temp = N \cdot \alpha$, $\alpha = 0.85$
		\item Chłodzenie: $temp = temp \cdot \beta $, $\beta = 0.85$
		\item Długość epoki: $epoch\_range = initial\_temp \cdot \delta = 0.7$
		\item Liczba iteracji: $max\_it = N \cdot \delta$, $\gamma = 0.7$
		\item Typ otoczenia: \texttt{SWAP}
		\item Rozwiązanie początkowe: oparte o MST
	\end{itemize}

\subsubsection*{Symulowane wyżarzanie}
	\begin{itemize}
		\item Długość listy tabu: $t\_size = N \cdot \alpha$, $ \alpha = 0.1$
		\item Liczba iteracji: $max\_it = N \cdot \beta$, $\beta = 0.2$
		\item Typ otoczenia: \texttt{SWAP}
		\item Wybór otoczenia: pełne
		\item Rozwiązanie początkowe: oparte o MST
	\end{itemize}

\begin{table}[h!]
    \centering
    \begin{tabular}{|c|c|c|c|c|c|c|}
        \hline
        \multirow{2}{*}{Przykład} & \multicolumn{2}{|c|}{$local\_search$ 1} & \multicolumn{2}{|c|}{$local\_search$ 2}  & \multicolumn{2}{|c|}{$local\_search$ 3}  \\
        \cline{2-7}
        & \multicolumn{2}{c|}{średnia} & \multicolumn{2}{c|}{średnia} & \multicolumn{2}{c|}{średnia} \\
        \cline{2-4} \cline{4-5} \cline{6-7}
        & liczba popraw & suma wag & liczba popraw & suma wag & liczba popraw & suma wag \\
        \hline
        xqf131 & 28.8 & 701.9 & 62.3 & 712.4 & 51.9 & 722.4 \\
        \hline
        xqg237 & 37.0 & 1483.7 & 130.5 & 1415.8 & 112.2 & 1445.9 \\
        \hline
        pma343 & 75.8 & 1650.3 & 201.9 & 1684.7 & 183.3 & 1717.1 \\
        \hline
        pka379 & 78.9 & 1716.3 & 224.7 & 1745.9 & 204.9 & 1742.7 \\
        \hline
        bcl380 & 62.7 & 1949.3 & 223.7 & 1917.5 & 189.2 & 1913.5 \\
        \hline
        pbl395 & 80.0 & 1771.5 & 227.8 & 1829.0 & 193.4 & 1876.2 \\
        \hline
        pbk411 & 81.6 & 1837.9 & 242.7 & 1888.5 & 203.6 & 2181.15 \\
        \hline
        pbn423 & 72.3 & 1873.9 & 249.0 & 1921.6 & 219.8 & 2081.5 \\
        \hline
        pbm436 & 97.85 & 2562.5 & 256.8 & 2312.0 & 217.4 & 2363.1 \\
        \hline
        xql662 & 122.6 & 3690.4 & 405.4 & 3813.3 & 349.1 & 3917.5 \\
        \hline
        xit1083 & 191.6 & 4312.2 & 693.7 & 4721.2 & 623.9 & 4826.1 \\
        \hline
        icw1483 & 278.8 & 5034.6 & 973.4 & 5301.2 & 834.5 & 5718.2 \\
        \hline
        djc1785 & 382.6 & 6733.8 & 1193.9 & 6872.1 & 1032.6 & 7764.5 \\
        \hline
        dcb2086 & 390.9 & 7454.2 & 2392.6 & 7457.8 & 1222.2 & 8888.3 \\
        \hline
        pds2566 & 457.3 & 8671.4 & 1742.9 & 8701.8 & 1526.0 & 10687.5 \\
        \hline
    \end{tabular}
\end{table}

\end{document}